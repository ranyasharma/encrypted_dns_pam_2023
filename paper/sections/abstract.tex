\begin{abstract}
Unencrypted DNS traffic between users and DNS resolvers can lead to privacy
and security concerns.  In response to these privacy risks, many browser
vendors have deployed DNS-over-HTTPS (DoH) to encrypt queries
between users and DNS resolvers.  
Today, many client-side
deployments of DoH, particularly in browsers, select between only a
few resolvers, despite the fact that many more encrypted DNS resolvers are
deployed in practice.
Unfortunately, 
if users only have a few choices of encrypted resolver, and only a few
perform well from any particular vantage point, then the privacy problems that
DoH was deployed to help address merely shift to a different set of third
parties. It is thus important to assess the performance characteristics of
more encrypted DNS resolvers, to determine how many options for encrypted DNS
resolvers users tend to have in practice.
In this paper, we explore the performance
of a large group of encrypted DNS resolvers supporting DoH by measuring DNS
query response times from global vantage points in North America, Europe, and
Asia.  Our results show that many non-mainstream
resolvers have higher response times than mainstream resolvers, particularly
for non-mainstream resolvers that are queried from more distant vantage
points---suggesting that most encrypted DNS resolvers are not replicated or
anycast.
In some cases, however, certain non-mainstream resolvers perform 
at least as well as mainstream resolvers, suggesting that
users may be able to use a broader set of encrypted DNS resolvers than those
that are available in current browser configurations.
\end{abstract}
