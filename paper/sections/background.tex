\section{Background and Related Work}\label{sec:background}

In this section, we provide background on encrypted DNS protocols, including
the current deployment status of encrypted DNS, as well as various related
work on measuring encrypted DNS.

\subsection{Background: Encrypted DNS}

The Domain Name System (DNS) translates human-readable domain names into
Internet Protocol (IP) addresses, which are used to route
traffic~\cite{rfc1035}.  These queries have typically been unencrypted, which
enables on-path eavesdroppers to intercept queries and manipulate responses.


\paragraph{Encrypted DNS.}
Protocols for encrypting DNS traffic have been deployed in recent years, including DNS-over-HTTPS (DoH) and DNS-over-TLS
(DoT).  Zhu et al. proposed DoT in 2016, which uses a dedicated port (853)
to communicate with resolvers over a TLS connection~\cite{zhu2015connection}. In contrast,
DoH---proposed by Hoffman et al. in 2018---establishes HTTPS sessions with
resolvers over port 443~\cite{rfc8484}.  This design decision enables
DoH traffic to use HTTPS as a transport, facilitating deployment as well as
making it difficult for network operators and eavesdroppers to intercept DNS
queries and responses~\cite{boettger2019empirical}. DoH can function in many
environments where DoT is easily blocked.

\paragraph{Moving the privacy threat.} Encrypting DNS queries and responses hides queries
from eavesdroppers but the recipient of the queries---the DNS resolver---can
see the queries~\cite{IEEEfight}. By design, recursive resolvers receive
queries from clients and typically need to perform additional queries to a
series of authoritative name servers to resolve domain names.  For these
resolvers to determine the additional queries they need to perform (or
determine if the query can be answered from cache), they must be able to see
the queries that they receive from clients.  Thus, although DoT and DoH make
it difficult for eavesdroppers along an intermediate network path to see DNS
traffic, recursive resolvers can still observe (and potentially, log) the
queries that they receive from clients.  The fact that many mainstream DoH
providers (e.g., Google) already collect significant information about users
potentially raises additional privacy concerns and makes it appealing for
users to have a large number of encrypted DNS resolvers that are reliable and
perform well. For this reason, users may wish to have more control over the
recursive resolver that they use to resolve encrypted DNS queries. Having
a reasonable set of choices that perform well in the first place is thus
important, and determining whether such a set exists is the focus of this
paper.

\subsection{Related Work}

\paragraph{Previous measurement studies of encrypted DNS.}
Previous studies have typically measured DoT and DoH response times the
protocols from the perspective of a few commonly used
resolvers~\cite{lu2019end-to-end}; in contrast, in this paper, we study a much
larger set of encrypted DNS resolvers, many of which are not available as
default options in major browsers. 

Hounsel et al. measure response times
and page load times for DNS, DoT, and DoH using Amazon EC2
instances~\cite{hounsel2020comparing}.  They compare the recursive resolvers
for Cloudflare, Google, and Quad9 to the local recursive resolvers provided by
Amazon EC2 from five global vantage points. They find that despite higher response times, page load times
for DoT and DoH can be \emph{faster} than DNS on lossy networks.  Lu et al.
utilized residential TCP SOCKS networks to measure response times from 166
countries and found that, in the median case with connection re-use, DoT and
DoH were slower than conventional DNS over TCP by 9 ms and 6 ms,
respectively~\cite{lu2019end-to-end}.
In contrast to previous work, our focus in this paper is not to measure the
DoH protocol itself or its relative performance to unencrypted DNS; instead,
our goal is to compare the performance of encrypted DNS resolvers {\em to
each other}, to understand the extent to which this larger set of DNS
resolvers could be used by clients and applications in different regions.