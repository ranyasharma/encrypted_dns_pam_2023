\section{Conclusion}\label{sec:conclusion}

The increased deployment of encrypted DNS, including DNS-over-HTTPS~(DoH) has
been accompanied both with ``mainstream'' DoH deployments in major browser
vendors, as well as a much broader deployment of encrypted DNS servers around
the world that are not among the common choices for resolvers in major
browsers.  Understanding the viability of this larger set of encrypted DNS
resolvers is important, particularly given that a lack of diversity of viable
resolvers potentially could create new privacy concerns, if only a small
number of organizations provided good performance. We find that many
non-mainstream resolvers have higher median response times than mainstream
ones, particularly if the resolvers are not local to the region; in contrast,
most mainstream resolvers appear to be replicated and provide better response
times across different geographic regions. In some cases, however, a
local non-mainstream resolver can exhibit equivalent performance as compared
mainstream resolvers (\eg, {\tt ordns.he.net}, {\tt freedns.controld.com},{\tt
dns.brahma.world}, and {\tt dns.alidns.com}). These results
suggest both good news and room for improvement in the future: On the one
hand, viable alternatives to mainstream encrypted DNS resolvers do exist.  On
the other hand, users need easy ways of finding and selecting these
alternatives, whose availability and performance may be more variable over
time than mainstream resolvers. It is also clear that there is an opportunity
to invest in deploying and maintaining reliable, performant, global encrypted
DNS infrastructure operated by a greater diversity of organizations.

